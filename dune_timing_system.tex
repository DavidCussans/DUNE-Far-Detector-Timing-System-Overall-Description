\documentclass{article}
%\documentclass[final]{dune}
\usepackage[utf8]{inputenc}

\usepackage{graphicx}
\usepackage{xcolor}

\usepackage{float}

\usepackage{xspace}

% if we aren't using the DUNE class then we need the packages...
\usepackage{siunitx}
% end of packages included instead of using DUNE

\input{units.tex}
\usepackage[hidelinks]{hyperref}
\input{defs.tex}
\input{glossary.tex}

\usepackage{biblatex}
\addbibresource{bibliography.bib}

\title{DUNE Far Detector Timing and Synchronization System}
\author{David Cussans, Stoyan Trilov}
\date{February 2020}

\begin{document}

\maketitle

\section{Introduction}
\label{sec:intro}
%Describe the generation of timing/synchronisation signals and and distribution to the detectors.
The primary functions of the \dword{dts} are to provide a stable and phase-aligned master clock to all DAQ components; receive external signals, which may include triggers, into the DUNE clock domain and time-stamp them; distribute synchronization, calibration, and potentially trigger commands to the DAQ system; conduct continuous checks of its own function; and provide status information to the DUNE \dword{ccm}. In addition, the timing system acts as a data source, providing a record of timing signals received, distributed, or throttled. The goals of the \dword{dune} physics programme impose the following synchronisation requirements:

\begin{itemize}
    \item synchronisation between components of a \dwords{detmodule} of $\mathcal{O}(\SI{1}{\ns})$
    \item synchronisation between \dwords{detmodule} of $\mathcal{O}(\SI{1}{\us})$ 
    \item synchronisation between \dword{dts} timestamp and \dword{utc} of $\mathcal{O}(\SI{1}{\ms})$.
\end{itemize}

Due to different technical requirements and development history of the \dword{sp} and \dword{dp} technologies, different clock frequencies and distribution methods are used for each one. A \dword{spmod} has many more timing endpoints than a \dword{dpmod} and many of the end points are simpler than the end points in the \dword{dpmod}, for example a \dword{wib} versus \dword{utca} crate.

Components of a \dword{dpmod} are synchronized to a \SI{125}{\MHz} clock distributed using a \dword{wr} network \cite{wr_ohwr}. Components of a \dword{spmod} are synchronized to a \SI{62.5}{\MHz} clock, which is distributed to \dword{spmod} endpoints using a 8b/10b encoded serial data stream. Both clocks are derived from a common \SI{10}{\MHz} reference provided by a GPS receiver, using a GPS disciplined oscillator. The GPS receiver also provides \dword{utc} signals used for generating and synchronising the \dword{dts} timestamp. The \dword{sp} and \dword{dp} clocks are locked together.

\section{System design}
\label{sec:system_design}
\subsection{Overview}
An FPGA-based \dword{gib} receives a high-quality \SI{10}{\MHz} clock, provided by a \dword{gpsdo}. The \dword{gib} also receives a serial \dword{irig} signal \cite{irig}, which has \dword{utc} encoded within it. Both the GPS receiver and the \dword{gib} are located on the surface. The \dword{gib} also receives external signals from the calibration system and the beam instrumentation. It interfaces to the DUNE \dword{ccm} and DAQ via a gigabit ethernet interface. The \dword{gib} multiplexes synchronization and trigger commands, along with arbitrary command sequences generated by software, into a single encoded data stream, which is transmitted to each \dword{spmod} using a single mode optical fibre.

The data stream at each \dword{spmod} is received by a \dword{mib}, and decoded into separate clock and data signals. The \dword{mib}, which is hosted in a \dword{utca} crate, distributes the recovered clock and data signals to six \dwords{amc}, also hosted in the same crate. The clock and timing data are fanned out to the \dwords{amc} over the \dword{utca} backplane. The \dwords{amc} are the carriers of the \dwords{fib}, which are used to broadcast the timing data stream to each timing endpoint, using single mode optical fibres. Each \dword{fib} has eight \dword{sfp} transceiver modules, where an \dword{sfp} serves two \dwords{apa}, through the use of an optical spliitter/combiner (assuming two timing endpoints per \dword{apa}).

The serial timing data stream is decoded at each endpoint into separate clock and data signals. A uniform phase-aligned cycle counter, updating at the SPDTS frequency of \SI{62.5}{\MHz}, is maintained at all endpoints, allowing commands to take effect simultaneously at all endpoints regardless of cable lengths or other phase delays. Figure \ref{fig:daq-readout-timing} shows the overall arrangement of components within the \dword{spdts}.

\begin{figure}[h]
\includegraphics[width=\textwidth]{DUNE_SP_Timing_DAQ_on_CryoM_GPS_In_CUC_16mar20.pdf}
\caption{Illustration of the components in the \dword{spdts}.}
\label{fig:daq-readout-timing}
\end{figure}

In order to provide redundancy, and also the ability to easily detect
issues with the timing path, two independent GPS systems, i.e. GPS receiver+\dword{gib}, are used. One with an antenna at the head of the Yates Shaft, the other with an
antenna at the head of the Ross Shaft. The two independent timing
paths are brought to each \dword{detmodule} cavern, where they both feed two independent redundant \dword{utca} crates, each equipped with a \dword{mib} and a set of six \dwords{amc}+\dwords{fib}. Using 2:1 fibre combiner/splitters, one \dword{spdts} unit can be left as a hot spare while the
other is active. This also allows testing of new firmware and software
during commissioning without the risk of losing the \dword{spdts} if a bug is
introduced.

\subsection{GPS receiver}
In order to meet the synchronisation requirements outlined in section \ref{sec:intro}, the \dword{dts} needs a stable timing reference from which to generate the \dword{sp} and \dword{dp} clocks. The reference clock must have both excellent short and long term stability. Such a clock can be provided by a \dword{gpsdo}, which works by locking the output of a high quality quartz crystal or rubidium oscillator to a GPS signal. A GPSDO combines the long term stability of a GPS signal, with the short term stability of an oscillator.

A GPS signal can also be used to provide an absolute time reference, as GPS time is locked to \dword{utc}. The time from the GPS signal will be used to initialise and cross-check the timestamp provided by the \dword{dts}.

The \dword{dts} is envisaged to use the SecureSync Time and Frequency Synchronization System \cite{secure_sync_datasheet} as its GPS receiver. The SecureSync system provides a \SI{10}{\MHz} clock generated by a GPS disciplined \dword{oxco}. The GPS time signal will be output by a dedicated module \cite{secure_sync_modules} using an IRIG time code \cite{irig}. IRIG time codes are a standardised format of transmitting timing information. The \SI{10}{\MHz} clock and \dword{irig} data stream will be transmitted to the \dword{gib} using BNC connections.

The GPS receiver will also host a module \cite{secure_sync_modules}, which will output an IEEE 1588-2008 \cite{ieee_1588_2008} timing signal to the \dword{dp} \dword{wr} \cite{wr_ohwr} timing network.

\subsection{GIB}
The \dword{gib} receives the \SI{10}{\MHz} reference from the GPS receiver, and uses it to generate the \SI{62.5}{\MHz} \dword{spdts} clock. The clock is generated using an SI5344 chip, enabling the highest level of jitter performance. The IRIG time code is decoded by the \dword{gib} FPGA, and used to initialise the \dword{spdts} timestamp. After initialisation, the \dword{spdts} timestamp is maintained by the \dword{gib} firmware, where it is continuously updated using the \SI{62.5}{\MHz} clock. The firmware also continuously monitors, and records, the relationship between \dword{utc} and the \dword{spdts} timestamp.

Synchronisation messages, containing the \dword{spdts} timestamp, are generated by the \dword{gib} and transmitted to the \dwords{mib} using the \dword{spdts} protocol over single mode optical fibres. Before transmission, the 8b/10b encoded \dword{spdts} data stream is re-clocked using a high-speed D flip-flop running at \SI{312.5}{\MHz}. The \SI{312.5}{\MHz} clock is derived from the same reference as the \dword{spdts} clock, and is generated by the same IC.

The optical fibre link to the \dword{mib} is bi-directional, where the outgoing and incoming signals are transmitted using different wavelengths, transmitting happens at \SI{1310}{nm}, whereas receiving is at \SI{1550}{nm}. The return data path will be used to measure the time delay between the \dword{gib} and the \dword{mib}, allowing the \dword{spdts} to compensate for it. The logical layout of the \dword{gib} is shown in figure \ref{fig:gib_layout}.

\begin{figure}[h]
\includegraphics[width=\textwidth]{gib_layout.pdf}
\caption{Illustration of the components in the \dword{gib}.}
\label{fig:gib_layout}
\end{figure}

Control and management of the \dword{gib} and GPS receiver will be done using a gigabit ethernet.
\subsection{MIB}
The \dword{mib} receives the \dword{spdts} data stream from the \dword{gib} and decodes it into separate clock and data signals using a commercial \dword{cdr} IC. The recovered clock is fed into an SI5344 chip to regenerate a low jitter \SI{312.5}{\MHz} clock. The regenerated clock is fanned out to the six \dwords{amc} over the \dword{utca} backplane using a clock fan-out chip. The recovered data is sent directly into the \dword{mib} FPGA. The upstream data stream, i.e. from the \dword{mib} to the \dword{gib}, is re-clocked using the low jitter \SI{312.5}{\MHz} clock, before being sent out over the optical fibre.

The \dword{utca} backplane provides three point-to-point differential connections between the \dword{mib} and each \dword{amc}. One of the links is used to fan out the \SI{312.5}{\MHz} clock to the \dwords{amc}. Another one is used to transmit \dword{spdts} timing data to each \dword{amc}. The third line serves as a return data path, from each \dword{amc} back to the \dword{mib}. An illustration of the different \dword{mib} components is shown in figure \ref{fig:mib_layout}.

\begin{figure}[h]
\includegraphics[width=\textwidth]{mib_layout.pdf}
\caption{Illustration of the components in the \dword{mib}.}
\label{fig:mib_layout}
\end{figure}

The \dword{mib} has the following functions:

\begin{itemize}
	\item Reception of external timing and trigger signals
	\item Logging and time-stamping of signals received, distributed or throttled, and transmission to DAQ
	\item Serialisation of timing commands and transmission to the timing network
	\item Phase measurement of incoming timing signals from slaves, i.e. \dwords{fib} and timing endpoints, allowing phase adjustment under software control
	\item Transmission of arbitrary commands and control data under software control
\end{itemize}

The reception of external signals will be achieved using the return data path from a dedicated timing endpoint connected to a \dword{fib}. The external signals will be timestamped onto \dword{spdts} time domain by the dedicated endpoint. The external signal \dword{spdts} message will then be sent from the endpoint, to the \dword{mib}, via a \dword{fib}. Depending on the type of external signal message, the \dword{mib} may redistribute it to all timing endpoints, or take another action. 
As well as timestamping external signals onto the \dword{spdts} time domain, the \dword{mib} will do the same for \dword{wr} synchronization signals from the \dword{dp} \dword{detmodule}. Reciprocally, \dword{spdts} synchronization signals will be timestamped onto the \dword{dp} clock domain. This allows the timing in the \dword{sp} and \dword{dp} \dwords{detmodule} to be
aligned. A similar scheme is used to relate the \dword{spdts} timing domain to the beam instrumentation \dword{wr} time domain.

To allow timing commands to take effect at all endpoints simultaneously, the delay between the command source, i.e.\ \dword{mib}, and each endpoint, must be measured and compensated for. To achieve this, both the \dword{mib}-\dword{fib}, and \dword{fib}-endpoint delays, must be ascertained.

The interface to the DUNE DAQ and \dword{ccm} systems will be implemented over gigabit ethernet. The \dword{utca} backplane will provide an ethernet connection between the \dword{mib} and the \dword{mch}, which in this case, acts as an
ethernet switch.

\subsection{FIB and its carrier}
The \dword{fib}, together with its carrier (an \dword{amc}), act as an active fanout for the \dword{spdts} clock and data stream. The clock delivered to the \dword{amc} over the \dword{utca} backplane is routed to the \dword{fib}, where it is fed into a SI5344 IC, generating a low jitter version of the clock. The incoming data is routed to the \dword{amc} FPGA, where it is decoded and also fanned out to the eight \dword{sfp} modules on the \dword{fib}. As before, the \dword{spdts} data is re-clocked on the low jitter version of the \SI{312.5}{\MHz} clock using a D-type flip-flop, before it is sent out.

The \dword{fib} is also used to arbitrate which of the eight \dword{sfp} modules is sending data upstream back to the \dword{mib}. This is achieved via an 8:1 differential multiplexer controlled by the \dword{amc} FPGA. \textcolor{red}{Is this true? how does this affect the external signals endpoint?} The output of each of the \dword{sfp} modules is connected to an 1:8 passive optical splitter/combiner, allowing each \dword{sfp} to service two \dwords{apa}, assuming four timing endpoints per \dword{apa}. The fact that multiple endpoints are connected to one \dword{sfp} means that only one endpoint can send \dword{spdts} data at a time. The endpoint receiving the external triggers will not be connected to an optical splitter/combiner. A block diagram of the components of the \dword{fib} and its carrier is shown in figure \ref{fig:fib_and_carrier_layout}.

\begin{figure}[h]
\includegraphics[width=\textwidth]{protoDUNE_timing_system_active_fanout.pdf}
\caption{An illustration of the \dword{fib} and its carrier. \textcolor{red}{This ia a diagram of the ProtoDUNE-SP I fanout unit, replace with actual diagram.}}
\label{fig:fib_and_carrier_layout}
\end{figure}

The \dword{fib} carrier is envisaged to be off-the-self \dword{amc}. The ohwr.org hosted \dword{afc} design \cite{amc_ohwr}, is currently being evaluated as a potential candidate.

The \dword{fib} will be a development of the existing timing \dword{fmc} used in ProtoDUNE-SP I. A photo of the existing timing FMC, and a mock-up of the \dword{fib} are shown in figures \ref{fig:timing_fmc} and \ref{fig:fib_mockup} respectively.

\begin{figure}[h]
\includegraphics[width=\textwidth]{timing_fmc.pdf}
\caption{A photo of the existing timing FMC.}
\label{fig:timing_fmc}
\end{figure}

\begin{figure}[h]
\includegraphics[width=\textwidth]{fib_mockup.png}
\caption{A mock-up of the \dword{fib}.}
\label{fig:fib_mockup}
\end{figure}

Monitoring and control of each \dword{amc} and its \dword{fib} will be done over gigabit ethernet. Similarly as before, the \dword{utca} backplane provides an ethernet connection between each \dword{amc} and the \dword{mch}, which will act as an ethernet switch.

\subsection{Endpoint}
Endpoints decode the timing signal into separate clock and data
signals using a \dword{cdr} IC, which in turn feeds a low-bandwidth PLL for applications requiring a very low jitter local clock. A common firmware block is used to decode the timing protocol, which is incorporated into the overall
firmware design for the receiving FPGA in each DAQ component. This
`block provides a cycle counter, several independent trigger, calibration and 
synchronisation signals, and a general-purpose packet data output to each endpoint. The cycle counter may be used further to generate low-frequency timing signals for further propagation, e.g. the \SI{2}{\MHz} sampling signal for the cold ADCs.
\subsection{Redundancy}
The extremely high uptime requirements of the DUNE experiment translate into similarly high uptime requirements on the timing system. The uptime requirements on the timing system may even exceed those of the overall experiment due to the fact that the timing system is an essential part of the experiment infrastructure, and no system can operate without it.

To facilitate the uptime requirements, the timing system has been designed to include two complete and independent timing chains, where the whole or a part of one system is active, and the other one acts as hot spare.

There are two GPS antennae, one on top of the Yates shaft and another on top of the Ross shaft. Each antennae feeds a GPS receiver and a \dword{gib}, producing two independent \dword{spdts} clocks and timestamps. Cross-referencing the two independent signals, potentially against a third signal, e.g. a rubidium oscillator, allows for any drifts or other issues in one of the \dword{spdts} timing streams to be detected. At any one time, only one of the \dword{spdts} clock and timestamp are actively used.

The two \dword{spdts} timing streams are both fed into two \dword{utca} crates. There is one set of two \dword{utca} crates per \dword{spmod}. Each crate is populated with the full timing \dword{utca} payload, i.e. a \dword{mib}, \dword{mch}, and six \dwords{amc} and \dwords{fib}. The output of the two crates is brought together into a passive optical splitter/combiner, allowing the swapping of active crates without physical intervention.

The design also allows, both crates to be partially active. This may be useful in a scenario, where a \dword{fib} in the active crates becomes faulty, and the recovery action would be to enable the equivalent \dword{fib} in the redundant crate. This would isolate the disruption of the timing signal only to the endpoints supplied by the faulty \dword{fib}, rather than the whole \dword{spmod}. It is also possible to use just a single \dword{sfp} module of the redundant crate, minimising disruption to the DUNE data-taking even further.

The arbitration of which \dword{spdts} timing stream and physical hardware is providing the timing for the \dword{spmod} will be done by the DUNE \dword{ccm}. The \dword{ccm} system is natural candidate for this role, as it has high-level overview of the whole DUNE detector, including the timing system. The CCM may initiate a hot-swap action, however the responsibility for the actual implementation and execution of the swap mechanism lies in the domain of the timing system.

\subsection{Internal interfaces}
This section seeks to describe the physical \textcolor{red}{(and later software+firmware?)} interfaces between the different components of the \dword{spdts} and \dword{dpdts}. 

\subsubsection{GPS receiver-\dword{dpdts}}
The GPS receiver houses a IEEE 1588 (PTP) grandmaster and a source of IRIG time codes. The PTP grandmaster provides a timing signal for the \dword{dp} \dword{wr} timing network. The \dword{dpmod} uses the \dword{wr} implementation of the IEEE 1588-2008 timing distribution standard. The components that distribute the timing signals are included in the scope of the \dword{dp} readout electronics and are not described here. The interface between the \dword{dts} and the \dword{dp} readout electronics is by means of 1000Base-BX \dword{sfp} coupled by single mode fibre. There will be two fibres carrying IEEE 1588-2008 timing signals supplied to the \dword{dp} readout electronics, one from each of the two GPS receivers.

\subsubsection{GPS receiver-\dword{gib}}
There are two distinct hardware interfaces between the GPS receiver and the \dword{gib}. One delivering the \SI{10}{\MHz} reference clock from the GPS receiver to the \dword{gib}, the other transmitting \dword{irig} time code, containing the \dword{utc} timestamp. Both interfaces are realised using coaxial cables and BNC connectors.

The two physical units are envisaged to be close to each physically, either in the same or adjacent computing racks. This means that the length of the cables connecting the two devices will not exceed a few metres.

\subsubsection{\dword{gib}-\dword{mib}}
Each \dword{gib} will need connect to two \dwords{mib} per \dword{spmod}, where each connection will use a single mode fibre. The \dword{gib} and \dword{mib} will have \dword{sfp} cages, which will host 1000Base-BX-20 \dword{sfp} transceivers which transmit at 1310 nm and receive on 1550 nm. Since the \dword{gib} will be on the surface, and the \dword{mib} underground, the optical cables will be several km long.

\subsubsection{\dword{mib}-\dword{amc}}
The \dword{mib}, will sit in the primary \dword{mch} slot of the \utca crate, and will use three differential point-to-point lines to transmit and receive timing data from the six \dwords{amc}. The connections are provided by the \dword{utca} backplane. The FCLKA and TCLKA lines from the \dword{mib} are routed to CLK1 and CLK2 on each \dwords{amc}. These lines will be used to transmit the \dword{spdts} clock and data stream respectively. CLK3 from the \dwords{amc} is routed to TCLKB on the \dword{mib}. Two differential lines are used to establish a 1000BASE-X ethernet connection between the \dwords{amc} and the actual \dword{mch}, residing in the secondary \dword{mch} slot. An illustration of the connections inside the \dword{utca} is shown in figure \ref{fig:mib_utca_connections}.

\begin{figure}[h]
\includegraphics[width=\textwidth]{mib_utca_connections.pdf}
\caption{An illustration of the connections between \dword{mib} and \dwords{amc} inside the \dword{utca} crate.}
\label{fig:mib_utca_connections}
\end{figure}

\subsubsection{\dword{amc}-\dword{fib}}
The connection between the \dword{fib}, which is an \dword{fmc}, and its \dword{amc} carrier will be achieved using a \dword{hpc} \dword{fmc} connector. The \dword{afc} has two such connectors, however since the \dword{fib} is a single double-width \dword{fmc}, only one will be used.
\subsubsection{\dword{fib}-endpoint}
The output of the 48 ($6\times8$) \dword{sfp} modules from the two \dword{utca} crates will be brought together into a 2:1 passsive optical combiner/splitter. This configuration allows the switching of operation between the two crates to happen without physical intervention. The output of the 2:1 optical combiner will be fed into a 1:8 splitter/combiner, which is used to fan out the \dword{spdts} timing stream to the endpoints. The optical connections will use single-mode fibres, and 1000Base-BX-20 \dword{sfp} transceivers, transmitting at 1310 nm and receiving on 1550 nm. Due to the use of the optical splitter to fan out the timing signal, only one endpoint at a time can transmit a return signal.
\subsection{\dword{spdts} protocol}
\textcolor{red}{description of the spdts protocol goes here}
\section{Design and system validation}
The development and validation of the \dword{dts} is based around iterative incremental model, where the functionality of the system is built up step-by-step. As per the overall DUNE development strategy, each system or technology  which will be used in the eventual DUNE \dword{fd} must first be demonstrated to be fit for purpose at the ProtoDUNE experiments at CERN. The timing system is no different, and has fully embraced this ethos. Two distinct \dword{spdts} prototypes will be used for the two phases of the ProtoDUNE SP experiment. The use of the \dword{spdts} at the ProtoDUNE experiments will be especially useful in validating the operational requirements, external system interfaces, and long-term stability of the system. The validation of the \dword{spdts} system will also heavily rely on test and measurements performed in a laboratory environment, e.g. alignment of timing endpoints to within \SI{1}{ns}. 

\subsection{Tests at ProtoDUNE-SP I}
A prototype of the \dword{spdts} was developed and successfully used to synchronise the components of ProtoDUNE SP I. The system uses an AIDA 2020 TLU as a master unit, which receives external signals from the trigger system and SPS accelerator, as well as a high-quality \SI{10}{\MHz} reference, which is used to generate the ProtoDUNE SP I \SI{50}{\MHz} clock. The master unit also multiplexes synchronization and trigger commands, along with arbitrary command
sequences generated by software, into a single encoded data stream. The timing data stream is fanned out to the endpoints via the means of an active fanout module and single-mode fibres. At each endpoint, a commercial \dword{cdr} device is used to recover separate clock and data signal from the data stream, which are used to synchronise the endpoint.

The system operated successfully for the duration of the data-taking and has demonstrated the following functionalities:

\begin{itemize}
    \item generation and distribution of a stable master clock to DAQ components
    \item timestamping of external signals onto the ProtoDUNE SP I time domain
    \item distribution of synchronisation, trigger and calibration commands to the DAQ system.
\end{itemize}

These functionalities form a core part of the expected role of the \dword{dts}, however certain other key aspects of the system were not tested at ProtoDUNE SP I. Some of these include:

\begin{itemize}
    \item GPS interface and IRIG decoding
    \item Synchronisation with UTC
    \item Synchronisation between endpoints
    \item \dword{utca} fanout hardware.
\end{itemize}

The issues listed above will be addressed by the prototype developed for ProtoDUNE SP II, as well as prototype testing in a laboratory.

\subsection{Tests at ProtoDUNE-SP II}
The ProtoDUNE-SP I timing system demonstrated several key functions of the eventual \dword{spdts}, however the hardware used to construct the prototype is not aligned with that of the actual DUNE \dword{spdts}. This issue be remedied by the ProtoDUNE-SP II prototype, which will use the full chain hardware components described in section \ref{sec:system_design}, i.e. from GPS receiver to timing endpoint. This second prototype will also be used to test synchronisation between endpoints and UTC. The validation of these requirements will be done both at ProtoDUNE SP II and in the laboratory. The frequency of the system clock will be increased to the expected \dword{spdts} value of \SI{62.5}{\MHz}.

Another advantage of deploying timing system prototypes at the ProtoDUNE experiments is that it allows the assessment of the long-term stability of the system, and associated operational challenges. This type of long-term operation feeds directly into the validation of the uptime requirements of the final \dword{dts}.

\subsection{Tests in laboratory}
For ProtoDUNE SP II, it is not envisaged that a fully redundant system will be deployed, however the functionalities associated with having two independent timing chains must still be developed and tested. Most of this work will be carried out in a laboratory setting, with a representative setup of the \dword{dts}. Laboratory tests will also be needed to carry out disruptive tests, such as endpoint or system wide fault recovery and firmware development.

\section{Commissioning}
The \dword{dts} will be one of the first DAQ components installed, so that timing and synchronization signals are available the other components of the DAQ as soon as they start to be installed. Early in the construction project \dword{spts} ``development kits'' will be made available. The kits will include the hardware and software needed to produce \dword{spts} timing signal. This will allow the teams developing the DUNE readout systems to integrate with the \dword{spts} early in the development process. Hardware and software will also be available for use in vertical slice tests and the \dword{itf}. 

\printglossary
\printbibliography

\end{document}
