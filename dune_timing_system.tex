\documentclass{article}
%\documentclass[final]{dune}
\usepackage[utf8]{inputenc}

\usepackage{graphicx}

\usepackage{float}

\usepackage{xspace}

% if we aren't using the DUNE class then we need the packages...
\usepackage{siunitx}
% end of packages included instead of using DUNE

\input{units.tex}
\usepackage[hidelinks]{hyperref}
\input{defs.tex}
\input{glossary.tex}
\title{DUNE Far Detector Timing and Synchronization System}
\author{David Cussans }
\date{October 2019}

\begin{document}

\maketitle

\section{Introduction}

%\subsection{Timing and Synchronization}
%\label{sec:fd-daq-timing}

% \metainfo{David Cussans \& Kostas Manolopoulos.  This is an SP-specific section.  It's file is \texttt{far-detector-single-phase/chapter-fdsp-daq/design-timing.tex}}

%Describe the generation of timing/synchronisation signals and and distribution to the detectors.

All components of the \dword{fd} are driven with clocks derived from a single GPS disciplined source. Components of the \dword{spmod} are
synchronized to a \SI{62.5}{\MHz} clock. Components of the \dword{dpmod} are synchronized to a \SI{125}{\MHz} clock. The two clocks are locked together.

In order make full use of the
information from the \dword{pds}, the common clock must be
aligned within a single \dword{detunit} with an accuracy of $O\,1$\si{ns}.
In order to form a common trigger for \dword{snb} between
\dwords{detmodule}, the timing between them must be aligned with an
accuracy of $O\,1$\si{ms}.  However, a tighter constraint is the need to
calibrate the common clock to universal time (derived from GPS) in
order to adjust the data selection algorithm inside an accelerator
spill, which requires an absolute accuracy of $O\,1$\si{\micro\s}.

\single and \dual \dwords{detmodule} use different timing systems,
driven by the different technical requirements and development history
of the two technologies. %systems. 
A \dword{spmod} has many more
timing end points than a \dword{dpmod} and many of the end points
are simpler than the end points in the \dual{}, for example a \dword{wib}
versus \dword{utca} crate. Both systems have been successfully prototyped and the DUNE-SP system has been sucessfully used at protoDUNE-SP.

\section{Single Phase}

The DUNE \dword{spmod} uses a development of the \dword{pdsp} timing system. 
Synchronization messages are transmitted over a serial
data stream with the clock embedded in the data. The format is
described in 
% DUNE DocDB-1651~\cite{docdb-1651}. 
DUNE DocDB-1651. 
Figure~\ref{fig:daq-readout-timing}
shows the overall arrangement of components within the \dword{spts}
%\single Timing System (SPTS). 
% \fixme{need a gloss term?}
A stable master clock, disciplined with a \SI{10}{\MHz}
reference is used in the \dword{spts}. An \dword{irig} signal is
also received by the system and used to set the \dword{spts} 64-bit time-stamp. However the periodic synchronization
messages distributed to the \dword{sp} \dword{detmodule} are an exact number
of clock cycles apart even if there is jitter in the \dword{irig} signal.
%\fixme{Need reference added for DocDB-1651.}  done - ATH, 4/16/18

The GPS signal is encoded onto optical fiber and transmitted to the
\dword{cuc}, where it is converted back to an RF signal on coaxial cable and
used as the input to a GPS disciplined oscillator. The oscillator module
also houses a IEEE 1588 (PTP) grandmaster and a source of IRIG-G time codes. The PTP grandmaster provides a timing signal for the \dword{dp} \dword{wr}
timing network. The IRIG-G time code provides an absolute time for the \dword{spts} by comparing the IRIG-G time code with the \dword{spts} time counter and reading.

The latency from the GPS antenna on the surface to the \dword{dts} master in
the \dword{cuc} will be compensated for by measuring the round trip delay over a single mode fibre in the same bundle at the raw GPS signal coming from the antenna. 

The \dword{wr} synchronization signals from the \dword{dp} \dword{detmodule} are
time-stamped onto the \dword{spts} clock domain and the \dword{spts} synchronization
signals are time stamped onto the \dword{dp} clock domain. This allows
the timing in the \dword{sp} and \dword{dp} \dwords{detmodule} to be
aligned. A similar scheme is used to relate the \dword{pdsp}
 timing domain to the beam instrumentation
\dword{wr} time domain.

In order to provide redundancy, and also the ability to easily detect
issues with the timing path, two independent GPS systems are used. One
with an antenna at the head of the Yates Shaft, the other with an
antenna at the head of the Ross Shaft. The two independent timing
paths are brought together in the same rack in the \dword{cuc}. Using 1:2
fibre splitters one \dword{spts} unit can be left as a hot spare while the
other is active. This also allows testing of new firmware and software
during commissioning without the risk of losing the \dword{spts} if a bug is
introduced.


%\begin{dunefigure}[Arrangement of components in DUNE timing system]{fig:daq-readout-timing}
%  {Illustration of the components in the DUNE timing system.}
%  % Generated using draw.io online tool from daq-timing-block-diagram-tdr_2nov18.html
%\includegraphics[width=0.8\textwidth]{daq-timing-block-diagram-tdr_2nov18.pdf}
%\end{dunefigure}

\begin{figure}[H]
  % Generated using draw.io online tool from daq-timing-block-diagram-tdr_2nov18.html
\includegraphics[width=0.8\textwidth]{daq-timing-block-diagram-tdr_2nov18.pdf}
\caption{Illustration of the components in the DUNE timing system.}
\label{fig:daq-readout-timing}
\end{figure}


All the custom electronic components for the \dword{spts} are contained in two
\dword{utca} shelves. At any one time one is active and the other is a
hot spare. The \SI{10}{\MHz} reference clock and the \dword{pps} are received
by a single width \dword{amc} at the center of the \dword{utca} shelf. This
master timing \dword{amc} is a custom board and produces the \dword{spts} signals and encodes them onto a
serial data stream. This serial datastream is distributed over a
standard star-point backplane to each of the fanout \dword{amc}. The fanout \dword{amc} is an off-the-self board, probably the ohwr.org hosted "AMC FMC Carrier" design, carrying two custom \dword{fmc}. Each \dword{fmc} has four SFP cages. The SFP cages are either occupied by
1000Base-BX SFPs, each of which connects to a fiber running to an \dword{apa},
or to a Direct Attach cable which connects to systems elsewhere in the
\dword{cuc}. This
arrangement is shown in Figure~\ref{fig:daq-readout-sp-timing}


%\begin{dunefigure}[Arrangement of components in \dlong{sp} timing system]{fig:daq-readout-sp-timing}
%  {Illustration of the components in the \single timing system.}
%  % Generated using draw.io online tool from DUNE_SP_Timing_1nov18.html
%  \includegraphics[width=0.8\textwidth]{DUNE_SP_Timing_1nov18.pdf}
%\end{dunefigure}

\begin{figure}[H]
  % Generated using draw.io online tool from DUNE_SP_Timing_1nov18.html
  \includegraphics[width=0.8\textwidth]{DUNE_SP_Timing_1nov18.pdf}
  \caption{Illustration of the components in the \single timing system.}
  \label{fig:daq-readout-sp-timing}
\end{figure}

\subsection{Commissioning}

The \dword{dts} will be one of the first DAQ components installed in the CUC so that timing and synchronization signals are available the other components of the DAQ as soon as they start to be installed. Early in the construction project \dword{spts} ``development kits'' will be made available. The kits will include the hardware and software needed to produce \dword{spts} timing signal. This will allow the teams developing the DUNE readout systems to integrate with the \dword{spts} early in the development process. Hardware and software will also be available for use in vertical slice tests and the \dword{itf}. 

\section{Dual Phase}

The \dword{dpmod} uses the \dword{wr} implementation of the IEEE1588-2008 timing distribution standard. The components that distribute the timing signals are included in the scope of the \dword{dp} readout electronics and are not described here. The interface between the \dword{dts} and the \dword{dp} readout electronics is by means of 1000Base-BX \dword{sfp} coupled by single mode fibre. There will be two fibres carrying IEEE1588-2008 timing signals supplied to the \dword{dp} readout electronics, one from each of the two GPS receivers.



\subsubsection{Beam timing}
\label{sec:fd-daq-design-beamtiming}

The neutrino beam is produced at the Fermilab accelerator complex in
spills of \SI{10}{\us} duration. 
A \dword{sls} at the Far Detector site will locate the time periods in
the data when beam could be present, based on network packets received
from Fermilab containing predictions of the GPS-time of spills soon to
occur or absolute time stamps of recent spills. % recently occurring. 
Experience from MINOS and \nova shows that this can provide beam
triggering with high reliability with some small fraction of late or
dropped packets.
To improve reliability further, the system outlined here contains an extra layer
of redundancy in the prediction process. 
Several stages of prediction based on recent spill behavior will be applied, aiming
for an accuracy of better than 10\% of a readout
time (sub-\si{\ms}) in time for the data to be selected from
the \dword{daq} buffers. 
Ultimately, an offline database will match the actual time of the
spill with the data, thus removing any reliance on real-time network
transfer for this crucial stage of the oscillation measurements. The
network transfer of spill-timing information is simply to ensure that a
correctly located and sufficiently wide window of data is considered
as beam data. This system is not required, and is not designed to
provide signals accurate enough to measure neutrino time-of-flight.

The precision to which the spill time can be predicted at \fnal
improves as the acceleration process of the protons producing the
spill in question advances.  The spills currently occur at intervals
of \SI{1.3}{\s}; the system will be designed to work with any interval, and
to be adaptable in case the sequence described here changes.  For
redundancy, three packets will be sent to the far detector for each
spill.  The first is approximately \SI{1.6}{\s} before the spill-time, which
is at the point where a \SI{15}{\Hz} booster cycle is selected; from this
point on, there will be a fixed number of booster cycles until the
neutrinos and the time is subject to a few \si{\ms} of jitter.  The second
is about \SI{0.7}{\s} before the spill, at the point where the main injector
acceleration is no longer coupled to the booster timing; this is
governed by a crystal oscillator and so has a few \si{\us} of jitter.
The third will be at the so called `\texttt{\$74}' signal generated before the beamline kicker magnet fires
to direct the protons at the LBNF target; this doesn't improve the
timing at the Far Detector much, but serves as a cross check for
missing packets.  This system is enhanced compared to that of
MINOS-\nova, which only use a third of the above timing signals.  The
reason for the larger uncertainty in the time interval from \SI{1.6}{\s} to
\SI{0.7}{\s} is that the booster cycle time is synchronized to the
electricity supply company's \SI{60}{\Hz}, which has a variation of about
1\%.

Arrival-time monitoring information from a year of MINOS data-taking
was analyzed, and it was found that 97\% of packets arrived within
\SI{100}{\ms} of being sent and 99.88\% within \SI{300}{\ms}.

The \dword{sls} will therefore have estimators of the GPS-times of
future spills, and recent spills with associated data contained in the
\dwords{daqbuf}. These estimators will improve in precision as
more packets arrive.  The \dword{daq} will use data in a wider window than
usual, if, at the time the trigger decision has to be made, the
precision is lower %less accurate 
due to missing or late packets.  From the
MINOS monitoting analysis, this %will 
expected to be very rare.



\printglossary

\end{document}
